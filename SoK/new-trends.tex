\section{Emerging Trends} \label{sect:new-trends}

The concept of verifiable resource accounting (VRA) was first formalized by Sekar and Maniatis~\cite{SM11}.
Here, verifiability aims to give cloud customers assurance about two questions:
\begin{itemize}
 \item Did I consume what I was charged?
 \item Should I have consumed what I was charged?
\end{itemize}
The first question is about the accuracy of the consumption vector, e.g., CPU time, in/out network bandwidth, storage, etc. The cloud provider should not be able to charge a customer with resources it did not expend on her behalf. The second question concerns itself with the efficiency of the provider's infrastructure, e.g., with respect to scheduling and provisioning. If the provider erroneously used 1 GB of memory for a task instead of 1 MB, it should arguably not able to pass on the cost of its inefficiency to its customer.

It was envisioned in [SM11] that verifiable resource accounting can be realized in the following operating model: the provider generates a consumption report describing what resources it thinks a customer has consumed for a task. The customer then is able to check either by herself or via an independent component or party the validity of the consumption report. 

\subsection{Cryptographic Audits} \label{sect:crypto-audits}


- cryptographic VRA solution can be an appealing immediate and cheaper alternative. In fact, an important point to note is that a cryptographic solution can achieve properties not achievable via TPM, and vice versa. For example, we can provide proofs-of-data-fetch, proofs-of-storage, proofs-of-retrievability, proofs-of-computation, etc., while a TPM-based solution can measure CPU usage and memory utilization. So both approaches can complement each other.

\subsection{Trusted Computing} \label{sect:trusted-computing}

Chen et al.~\cite{CMP+13} proposed an instantiation of a verification mechanism via trusted trusted computing components. 
Their solution leverages advances in nested virtualization to build a trusted ``observer'' for monitoring and reporting resource use, e.g., CPU and memory usage.

They highlighted that trusted computing can be used to address resource accounting guarantees. However, this cannot be achieved using existing deployed mechanisms. Ideally, a new resource accounting OS or hypervisor should be developed. However, this is not viable given the existing legacy of deployed infrastructure. So, as shown in~\cite{CMP+13}, the only option is to build another layer of lightweight hypervisor on top of existing legacy hypervisors. Nevertheless, there are other challenges remain. For example, such a software-based solution currently can only monitor CPU usage and memory utilization. Also, there's a privacy issue with respect to the cloud provider. It's not clear how the provider's privacy can be protected in terms of their logic and policy of resource allocation (which is usually proprietary).

- the authors acknowledged that what can be done with current TPM technology is only a first step towards an overaching vision. So it may take years to see any deployable and satisfactory TPM-based solution for VRA.

\subsection{Others} \label{sect:other-trends}


~\cite{HDT+14,ZYS+14}

In THEMIS~\cite{PHC+13}, a trusted third party acting as a notary between the client and the cloud provider is used to verify the client’s resource consumption. It relies on digital signatures to protect the integrity and non-repudiability of billing transactions associated with resources consumed by the client. Billing statements issued by the cloud provider is compared against information obtained via a SLA monitoring module enhanced with TPM technology.

A peer-to-peer billing and accounting system for the cloud called BitBill was recently proposed~\cite{CC14} to remove assumptions relying on a trusted component or third party for verification purposes. It adopts a Bitcoin-like decentralized verification mechanism. That is, the trusted third party is replaced by a global log of all the resource provision and usage of all clients and providers. Every billable event has to be signed and broadcast, analogous to that of Bitcoin. The clients and providers verify and agree on a log of events with the help of other clients from the same resource pool. To prevent malicious nodes from forging false events, a similar proof-of-work technique is used such that the rate of announcement can be controlled.

- automating detection of SLAs violations~\cite{ENC+12,EBM+13}

\section{Current Landscape} \label{sect:current-practice}

\subsection{Cloud Service Level Agreements} \label{sect:sla-overview}

- Why are SLAs important?

~\cite{Ahr+10,FMM+13,Kyr13}

\begin{itemize}
 \item Resource availability: Cloud providers typically deliver their services via massively redundant systems to ensure high availability of their services.
 
 \item Data privacy: Data encryption, retention, and deletion. Many organizations have legal requirements that data must be kept for a certain period of time. Some organizations also require that data be deleted after a certain period of time.
 
 \item Hardware erasure and destruction: Improper disposal of hardware may lead to data leakage. If a hard drive fails, the platters of that disk should be zeroed out before the drive is disposed or recycled.

 \item Location: Depending on the kind of data the enterprise is managing on the user's behalf, there might be legal restrictions on the location of the physical server where the data is stored. Cloud vendors can provide an API for determining the location of the physical hardware that delivers the cloud service~\cite{Ahr+10}. For example, many countries prohibit storing personal information about its citizens on any machine outside its borders.
 
 \item Monitoring: The failure to meet the terms of an SLA has financial or legal consequences. So it is vital for the consumer to be able to monitor the performance of the provider. Arguably, the ideal solution has been to let a neutral third-party organization to perform monitoring tasks. This eliminates the conflicts of interest that might occur if the provider report outages at their sole discretion or if the consumer is responsible for proving that an outage occurred. (See \S\ref{sect:sla-monitoring} for discussion on various monitoring techniques and tools.)
 
 \item Auditability: The consumer is liable for any breaches that occur, and thus, it is  vital that the consumer be able to audit the provider's systems and procedures. However, audits are disruptive and expensive. The provider will most likely place limits and charges on them.
 
 \item Maintenance: Cloud providers should inform their consumers about how and when maintenance tasks are carried out.

 \item Resource accounting:
\end{itemize}

- lifecycle of applications and documents, including versioning of applications, the retention and destruction of data 


- we focus on SLA data policies~\cite{Mee12}: data preservation, data redundancy, data location, data privacy.



- Accounting \& Pricing Models~\cite{MMS13}


- currently lack of effective means for a cloud user to verify committed SLAs

\begin{table*}[htb]\centering \footnotesize
\caption{SLA requirements and use case scenarios~\cite{Ahr+10}.}
\label{tab:model-and-requirements}
  \begin{tabular}{lcccccc}
    \toprule
    & $\mathsf{U} \leftrightarrow \mathsf{C}$ & $\mathsf{E} \leftrightarrow \mathsf{C} \leftrightarrow \mathsf{U}$ & $\mathsf{E} \leftrightarrow \mathsf{C}$ & $\mathsf{E} \leftrightarrow \mathsf{C} \leftrightarrow \mathsf{E}$ & Private cloud & Hybrid cloud \\
    \midrule
    Data encryption &&& \cmark &&& \\
    Privacy & \cmark & \cmark & \cmark & \cmark & \cmark & \cmark \\
    Data retention \& deletion &&& \cmark & \cmark && \cmark \\
    Hardware erasure \& destruction &&& \cmark & \cmark && \cmark \\
    Regulatory compliance & \cmark & \cmark & \cmark & \cmark & \cmark & \cmark \\
    Transparency & \cmark & \cmark & \cmark & \cmark & \cmark & \cmark \\
    Certification & \cmark & \cmark & \cmark & \cmark & \cmark & \cmark \\
    Terminology for PKIs &&& \cmark & \cmark & \cmark & \cmark \\
    Metrics & \cmark & \cmark & \cmark & \cmark & \cmark & \cmark \\
    Auditability & \cmark &&&&&\\
    Monitoring & \cmark & \cmark & \cmark & \cmark & \cmark & \cmark \\
    Machine-readable SLAs &&&& \cmark && \\
   \bottomrule
   \multicolumn{7}{l}{\scriptsize Notation: $\mathsf{U}$ denotes end user; $\mathsf{E}$ denotes enterprise; $\mathsf{C}$ denotes cloud; $\leftrightarrow$ denotes interactivity.}
  \end{tabular}
\end{table*}



\begin{table}[htb]\centering \footnotesize
\caption{SLA comparison between cloud providers~\cite{Bas12}.}
\label{tab:sla-comparison}
  \begin{tabular}{lccc}
    \toprule
    & Amazon & Azure & Rackspace \\
    \midrule
    Service guarantee &&& \\
    Scheduled maintenance &&& \\
    OS/software patches &&& \\
    Service guarantee time period &&& \\
    Service credit &&& \\
    Violation reporting onus &&& \\
    Violation incident reporting &&& \\
    Violation claim filing &&& \\
    SLA publish date &&& \\
   \bottomrule
  \end{tabular}
\end{table}


\subsection{Monitoring} \label{sect:sla-monitoring}

- service level management via monitoring and measuring the performance of the provided services are key to determining if a SLA is met.
- SLM allows a cloud provider to make decisions based on its business objectives and technical realities. For example, the cloud provider could reallocate bandwidth or bring more physical hardware online when the throughput for a service is not meeting a consumer's requirements.
- From the consumer's perspective, SLM helps in making decisions (possibly in an automated manner) about the way it uses cloud services~\cite{Ahr+10}.


- motivations for monitoring~\cite{DLN12,EFN+12}

- review of existing cloud service monitoring tools~\cite{ABD+13,FEH+14}

- monitoring metrics: throughput, load balancing, elasticity, agility, customer service response time, etc. These are not difficult to measure.

\begin{table*}[htb]\centering \footnotesize
\caption{Existing commercial and open-source cloud monitoring tools and their key properties \& features~\cite{FEH+14}.}
\label{tab:monitoring-tools}
  \begin{tabular}{lccccccc}
     \toprule
     & Amazon & AzureWatch~\cite{AzureWatch} & Rackspace & CA Nimsoft & Monitis~\cite{Monitis} & Nagios~\cite{Nagios} & PCMONS~\cite{Pcmons} \\
     & CloudWatch~\cite{CloudWatch} && Cloud Monitor~\cite{RackspaceCloudMonitoring} & Monitor~\cite{Nimsoft} &&& \\
     \midrule
     Scalability & \cmark & \cmark & \cmark & \cmark &\cmark && \\
     Portability &&& Limited & \cmark & \cmark & Limited & \\
     Multi-tenancy & \cmark & \cmark & \cmark & \cmark && \cmark & \\
     Interoperability &&&&&&& \cmark \\
     Customizability & \cmark & \cmark & \cmark & \cmark & \cmark & \cmark & \cmark \\
     Resource usage metering & \cmark & \cmark & \cmark & \cmark & \cmark & \cmark & \cmark \\
     {\bf Verifiable measuring} &&&&&&& \\
     Service load monitoring & \cmark & \cmark & \cmark & \cmark & \cmark & \cmark & \cmark \\
     {\bf Service PKI monitoring} &&&&&&& \\
     Risk assessment & \cmark & \cmark & \cmark & \cmark & Limited & \cmark & \cmark \\
     Security breach monitoring & \cmark & \cmark & \cmark & \cmark && \cmark & \\
    \bottomrule
  \end{tabular}
\end{table*}

\subsection{Limitations} \label{sect:sla-limitations}

Existing major cloud providers (e.g., Amazon, Rackspace, and Microsoft) leave the burden of detecting SLA violation to the customer~\cite{Bas12}.

- verify based on well-defined metrics, Service Measurement Index (SMI)

CloudProof~\cite{PLM+11}


- Most public cloud services offer a non-negotiable SLA. With these providers, a consumer whose requirements are not met can either: (i) accept a credit towards next month's bill, or (ii) stop using the service.

- To provide services cost-effectively, a cloud will manage the pool of hardware resources for resource efficiency; one of the strategies that a cloud provider employs during periods of reduced consumer demand is to power off unused components. Whether for power management, or for hardware refresh, migration of customer workloads (data storage and processing) from one physical computer to another physical computer is a key strategy that allows a provider to refresh hardware or consolidate workloads without inconveniencing consumers~\cite{BGP+12}.



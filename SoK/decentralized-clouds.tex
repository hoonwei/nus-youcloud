\section{Towards Decentralized Clouds} \label{sect:decentralized-clouds}

The concept of cloud computing~\cite{AFG+10}, in large part, is already a reality today.
Computing resources are typically provided by large data centers owned and monopolized by only a few organizations, e.g., Amazon, Microsoft, and Rackspace, and made available to users via a distributed network.
Such {\em centralized clouds} may be easier to manage; they are, however, costly to maintain, e.g., high utility costs and security \& physical vulnerabilities.
Moreover, they may have difficulty in keeping pace with today's and future data growth.
They may also be bandwidth bottlenecks for applications requiring substantial scalability and elasticity~\cite{symform-slide,techrepublic}.

{\em Decentralized clouds} are emerging as attractive alternatives in recent years.
Instead of relying on large and centralized data centers, users harvest the excess computing resources available from Internet-connected computers distributed across the world; in return, the owners of the involved computers earn credits or money (typically in the form of digital currency) proportionate to their contributions.
This greatly reduces operation and maintenance costs in comparison with the centralized model.
A few notable examples of such commercial decentralized clouds are Storj~\cite{Storj}, Symform~\cite{Symform}, and MaidSafe~\cite{MaidSafe}.
We note that a decentralized cloud is beyond a multi-cloud, which combines existing centralized clouds into a single virtual cloud storage (e.g., MultCloud~\cite{MultCloud} and Syndicate~\cite{Syndicate}), or a hybrid-cloud, which typically refers to a combination of public and private clouds.
The basic idea of a decentralized cloud here is essentially the same as that of what was traditionally known as {\em volunteer computing} (e.g., SETI@home~\cite{Seti@home} and Folding@home~\cite{Folding@home}) and {\em desktop grids}~\cite{CKB+07} (e.g., SZTAKI Desktop Grid~\cite{sztaki} and EDGeS~\cite{edges}).
However, a decentralized cloud typically relies on a pricing model, e.g., pay per usage, and the associated accounting and billing services, analogous to those for the existing centralized clouds.
(See~\cite{DP12,CWH13,MKK13} for further examples and discussions on centralized vs.\ decentralized clouds, and volunteer vs.\ cloud computing.)

\subsection{Problem Definition} \label{sect:problem}

We consider a data owner (client) storing her data in encrypted form on a server, possibly residing on a centralized cloud or a peer-to-peer (P2P) network.
The owner may run computational intensive tasks over the encrypted data.
Instead of performing tasks on the server itself, the owner delegates them to multiple workers residing on a decentralized cloud. 
Each worker fetches its portion of the required dataset from the server, performs the requested computation, and returns the results to the server.
The owner can then, at her discretion, download and verify the results.
The worker is financially rewarded if the verification of the returned results passes through.
This is illustrated in Figure~\ref{fig:model}.
In this work, we ask the question of {\em how can the owner securely and efficiently verify the resource consumption of each worker without relying on a trusted third party or any trusted hardware component}?

\begin{figure}[h!]\centering
  \includegraphics[scale=0.30]{model.pdf}
  \caption{Outsource of data and computation to a decentralized cloud.}
  \label{fig:model}
\end{figure}

Our decentralized cloud setting has the following distinctive characteristics in comparison to centralized cloud settings previously considered:
\begin{itemize}
 \item A data owner distributes her datasets and delegate computational tasks to a decentralized and heterogeneous computing environment.
 \item There exists an intermediate party, i.e., application/data server, which facilitates delegation and distribution of datasets and computational tasks.
 \item The owner can dynamically update her datasets stored on the server, but the data delegated to workers is assumed to be static.
 \item The owner's dataset is stored on the server over an extended period of time, but any data distributed to the worker is only temporarily stored and deleted upon the completion of the assigned task.
\end{itemize}

We note that the introduction of a centralized entity to facilitate distribution of tasks is necessary in our decentralized infrastructure.
In fact, BOINC~\cite{And04} also relies on centralized data and application servers to distribute tasks within a volunteer computing network, and BitTorrent~\cite{Coh03} uses a trusted, centralized tracker to coordinate activities within a P2P network.

\paragraph{Threat Model.}
However, each worker is assumed to be untrusted and they may deviate from the intended computation for various reasons, e.g., to save on computational cost. 
%It is essential, therefore, for the client to be able to verify the correctness of the computation. 
On the other hand, the worker may not trust the data owner either, in the sense that the worker may not be rewarded even if it has completed the computation and proved the correctness of the computation.


\subsection{Challenges} \label{sect:challenges}

- it seems infeasible to assume or require each distributed computer in a decentralized cloud model to be equipped with TPM


\paragraph{Proofs of Data Fetch.}
- One challenge is how to reduce the number of verifications required over multiple workers (currently a prover can convince a verifier with a constant number of challenges/responses for each file). 

- Another issue is that if the worker knows in advance the positions of the challenge data blocks, then it simply downloads only those blocks. Existing POR systems do not encounter such a problem because a file is stored for a relatively extended period of time and different challenges on different positions may be issued to the worker. This forces the worker to download the entire file.

- During update of elements of the dataset, how can the owner update her local state information (used for verifying resource consumption) in an efficient manner?

\paragraph{Proofs of Computation.}
- why POC is sufficient for some applications compared to checking the correctness of computation? (However, for some applications, it may be easier to verify the correctness of computation, which could imply both PDF and POC.)





